%%=============================================================================
%% Vergelijking
%%=============================================================================

\chapter{\IfLanguageName{dutch}{Vergelijking services en tools}{Comparison services and tools}}
\label{ch:vergelijking}

In dit hoofdstuk worden de~\acrshort{iac} tools geëvalueerd en vergeleken op basis van de criteria in Hoofdstuk~\ref{sec:criteria-evaluatie}.
De tools worden in drie categorieën verdeeld:

\begin{itemize}
    \item Declaratieve tools
    \item Tools gebaseerd op programmeertalen
    \item Interactieve tools
\end{itemize}

% TODO: beschrijf de 3 categorieën


\section{Interactieve tools}
\label{sec:interactieve-tools}

Een Vereenvoudiging service of tool is een intuïtieve console of een \acrshort{cli} die het hosten van toepassingen op AWS-infrastructuur of het opzetten van AWS-resources aanzienlijk vereenvoudigt.
De volgende tools en services bijhoren tot deze categorie:

\begin{itemize}
    \item AWS AppRunner
    \item AWS Amplify
    \item AWS Lightsail
    \item AWS Copilot
    \item AWS Elastic Beanstalk
\end{itemize}

\subsection{Initiële inzet}
\label{subsec:initiële-inzet-ver}

Om deze tools te gebruiken is een basis kennis van cloud-infrastructuur (\acrshort{aws}) voldoende.
De gebruiker is meestal een ontwikkelaar die een toepassing op AWS wilt inzetten en het cloud-infrastructuur aspect wenst te beperken.
Bovendien zijn deze tools zo opgesteld, dat de gebruiker op een intuïtieve manier mee kan werken.
Daarom zijn speciale trainingen of opleidingen om deze tools in gebruik te nemen overbodig.

De documentatie is heel

\begin{table}[h!]
    \centering
    \caption{Interactieve tools - Initiële inzet}
    \makebox[\linewidth]{
        \begin{tabular}{| c || c | c | c |}
            \hline
            \multicolumn{4}{|c|}{Initiële inzet} \\
            \hline
            Tool                  & Documentatie & Training & Kost \\
            \hline
            AWS AppRunner         & +            & +        & +    \\\hline
            AWS Amplify           & +            & +        & +    \\\hline
            AWS Lightsail         & +            & +        & +    \\\hline
            AWS Copilot           & +            & +        & +    \\\hline
            AWS Elastic Beanstalk & +            & +        & +    \\\hline
        \end{tabular}
        \label{tab:inter-tools-inzet}
    }
\end{table}

\subsection{Operationele effectiviteit}
\label{subsec:operationele-eff-ver}


\begin{table}[h]
    \centering
    \caption{Interactieve tools - Operationele effectiviteit}
    \makebox[\linewidth]{
        \begin{tabular}{| c || c | c | c | c | c |}
            \hline
            \multicolumn{6}{|c|}{Operationele effectiviteit} \\
            \hline
            Tool                  & Procesfuncties & Maturiteit & Onderhoudbaarheid & Veiligheid & Categorisatie \\
            \hline
            AWS AppRunner         & +              & +          & +                 & +          & +             \\\hline
            AWS Amplify           & +              & +          & +                 & +          & +             \\\hline
            AWS Lightsail         & +              & +          & +                 & +          & +             \\\hline
            AWS Copilot           & +              & +          & +                 & +          & +             \\\hline
            AWS Elastic Beanstalk & +              & +          & +                 & +          & +             \\\hline
        \end{tabular}
        \label{tab:inter-tools-op}
    }
\end{table}

\subsection{Gebruiksgemak}
\label{subsec:gebruiksgemak-ver}

\begin{table}[h]
    \centering
    \caption{Interactieve tools - Gebruiksgemak}
    \makebox[\linewidth]{
        \begin{tabular}{| c || c | c | c | c | c |}
            \hline
            \multicolumn{6}{|c|}{Gebruiksgemak} \\
            \hline
            Tool                  & Herbruikbaarheid & Ready-made & Cross-platform & Syntax & Open-source \\
            \hline
            AWS AppRunner         & +                & +          & +              & +      & +           \\\hline
            AWS Amplify           & +                & +          & +              & +      & +           \\\hline
            AWS Lightsail         & +                & +          & +              & +      & +           \\\hline
            AWS Copilot           & +                & +          & +              & +      & +           \\\hline
            AWS Elastic Beanstalk & +                & +          & +              & +      & +           \\\hline
        \end{tabular}
        \label{tab:inter-tools-gebruik}
    }
\end{table}

\subsection{Flexibiliteit}
\label{subsec:flexibiliteit-ver}

TEXT TEXT TEXT TEXT

\begin{table}[h!]
    \centering
    \caption{Interactieve tools - Flexibiliteit}
    \makebox[\linewidth]{
        \begin{tabular}{| c || c | c | c | c |}
            \hline
            \multicolumn{5}{|c|}{Flexibiliteit met} \\
            \hline
            Tool                  & Bestaande omgeving & IaC tools & DevOps tools & Development frameworks \\
            \hline
            AWS AppRunner         & +                  & +         & +            & +                      \\\hline
            AWS Amplify           & +                  & +         & +            & +                      \\\hline
            AWS Lightsail         & +                  & +         & +            & +                      \\\hline
            AWS Copilot           & +                  & +         & +            & +                      \\\hline
            AWS Elastic Beanstalk & +                  & +         & +            & +                      \\\hline
        \end{tabular}
        \label{tab:inter-tools-flex}
    }
\end{table}


\section{Tools gebaseerd op programmeertalen}
\label{sec:programmatieve-tools}

TEXT TEXT

\begin{itemize}
    \item AWS CDK
    \item Troposphere
    \item Pulumi
\end{itemize}

\subsection{Initiële inzet}
\label{subsec:initiële-inzet-pro-decl}

\begin{table}[h!]
    \centering
    \caption{Tools gebaseerd op programmeertalen - Flexibiliteit}
    \makebox[\linewidth]{
        \begin{tabular}{| c || c | c | c |}
            \hline
            \multicolumn{4}{|c|}{Initiële inzet} \\
            \hline
            Tool        & Documentatie & Training & Kost \\
            \hline
            AWS CDK     & +            & +        & +    \\\hline
            Troposphere & +            & +        & +    \\\hline
            Pulumi      & +            & +        & +    \\\hline
        \end{tabular}
        \label{tab:pro-tools-inzet}
    }
\end{table}

\subsection{Operationele effectiviteit}
\label{subsec:operationele-eff-pro}

\begin{table}[h!]
    \centering
    \caption{Tools gebaseerd op programmeertalen - Operationele effectiviteit}
    \makebox[\linewidth]{
        \begin{tabular}{| c || c | c | c | c | c |}
            \hline
            \multicolumn{6}{|c|}{Operationele effectiviteit} \\
            \hline
            Tool                  & Procesfuncties & Maturiteit & Onderhoudbaarheid & Veiligheid & Categorisatie \\
            \hline
            AWS AppRunner         & +              & +          & +                 & +          & +             \\\hline
            AWS Amplify           & +              & +          & +                 & +          & +             \\\hline
            AWS Lightsail         & +              & +          & +                 & +          & +             \\\hline
            AWS Copilot           & +              & +          & +                 & +          & +             \\\hline
            AWS Elastic Beanstalk & +              & +          & +                 & +          & +             \\\hline
        \end{tabular}
        \label{tab:pro-tools-op}
    }
\end{table}

\subsection{Gebruiksgemak}
\label{subsec:gebruiksgemak-pro}

\begin{table}[h!]
    \centering
    \caption{Tools gebaseerd op programmeertalen - Gebruiksgemak}
    \makebox[\linewidth]{
        \begin{tabular}{| c || c | c | c | c | c |}
            \hline
            \multicolumn{6}{|c|}{Gebruiksgemak} \\
            \hline
            Tool                  & Herbruikbaarheid & Ready-made & Cross-platform & Syntax & Open-source \\
            \hline
            AWS AppRunner         & +                & +          & +              & +      & +           \\\hline
            AWS Amplify           & +                & +          & +              & +      & +           \\\hline
            AWS Lightsail         & +                & +          & +              & +      & +           \\\hline
            AWS Copilot           & +                & +          & +              & +      & +           \\\hline
            AWS Elastic Beanstalk & +                & +          & +              & +      & +           \\\hline
        \end{tabular}
        \label{tab:pro-tools-gebruik}
    }
\end{table}

\subsection{Flexibiliteit}
\label{subsec:flexibiliteit-pro}

\begin{table}[h!]
    \centering
    \caption{Tools gebaseerd op programmeertalen - Flexibiliteit}
    \makebox[\linewidth]{
        \begin{tabular}{| c || c | c | c | c |}
            \hline
            \multicolumn{5}{|c|}{Flexibiliteit met} \\
            \hline
            Tool                  & Bestaande omgeving & IaC tools & DevOps tools & Development frameworks \\
            \hline
            AWS AppRunner         & +                  & +         & +            & +                      \\\hline
            AWS Amplify           & +                  & +         & +            & +                      \\\hline
            AWS Lightsail         & +                  & +         & +            & +                      \\\hline
            AWS Copilot           & +                  & +         & +            & +                      \\\hline
            AWS Elastic Beanstalk & +                  & +         & +            & +                      \\\hline
        \end{tabular}
        \label{tab:pro-tools-flex}
    }
\end{table}


\section{Declaratieve tools}
\label{sec:declaratieve-tools}

\begin{itemize}
    \item AWS Cloudformation
    \item AWS SAM
    \item Terraform
    \item Ansible
\end{itemize}

\subsection{Initiële inzet}
\label{subsec:initiële-inzet-decl}

\subsection{Operationele effectiviteit}
\label{subsec:operationele-eff-decl}

\subsection{Gebruiksgemak}
\label{subsec:gebruiksgemak-decl}

\subsection{Flexibiliteit}
\label{subsec:flexibiliteit-decl}


% Verdeling in 3 groepen declaratieve tools - CLI/wizard/ - programmatieve tools

%\begin{table}[h]
%    \label{tab:inzet-2}
%    \begin{tabular}{  l p{3.4cm}  p{3.4cm} p{3.4cm} }
%        \toprule
%        \textbf{Tool}      & \textbf{Documentatie} & \textbf{Training} & \textbf{Kost} \\\midrule
%        AWS Cloudformation & test                  & test              & test          \\\hline
%        AWS Terraform      & test                  & test              & test          \\\hline
%        AWS SAM            & test                  & test              & test          \\\hline
%        \bottomrule
%    \end{tabular}
%    \caption{Initiële inzet declaratieve tools}
%\end{table}
%
%\begin{table}[h]
%    \label{tab:inzet-3}
%    \begin{tabular}{  l p{3.4cm}  p{3.4cm} p{3.4cm} }
%        \toprule
%        \textbf{Tool} & \textbf{Documentatie} & \textbf{Training} & \textbf{Kost} \\\midrule
%        AWS CDK       & test                  & test              & test          \\\hline
%        Troposphere   & test                  & test              & test          \\\hline
%        Pulumi        & test                  & test              & test          \\\hline
%        \bottomrule
%    \end{tabular}
%    \caption{Initiële inzet programmatieve tools}
%\end{table}

%\begin{table}[h]
%    \label{tab:inzet-1}
%    \begin{tabular}{ l p{10cm} }
%        \toprule
%        \textbf{Tool/Service} & \textbf{Kost} \\\midrule
%        AWS AppRunner &
%            {\small
%        De kost voor een geprovisioneerde container instantie is \$ 0.007/GB-uur.
%        Voor een actieve container instantie is de kost \$ 0.064/vCPU-uur en \$ 0.007/GB-uur.
%        Daarnaast zijn er optionele add-ons en extra kosten als de toepassing andere AWS-services gebruikt.
%        } \\\hline
%        AWS Amplify &
%            {\small
%        Voor het gebruik van de Amplify Framework (libraries, CLI, UI Components) betaalt de gebruiker enkel voor de onderliggende AWS-resources.
%        De kosten voor het hosten van een statische webtoepassing zijn de volgende:
%        \begin{itemize}
%            \item \$ 0.01 per build minuut
%            \item \$ 0.023 per GB per maand
%            \item \$ 0.15 per GB
%        \end{itemize}
%        Publieke SSL-certificaten inclusief.
%        } \\\hline
%        AWS Lightsail &
%            {\small
%        Met Lightsail is de kost laag en voorspelbaar.
%        Deze service bundelt resources zoals geheugen, vCPU en SSD in een plan.
%        Alle plannen includeren een statisch IP-adres, \acrshort{dns}, \acrshort{ssh}-terminaal toegang, \acrshort{ssd}-opslag en monitoring.
%        } \\\hline
%        AWS Copilot &
%            {\small
%        Er zijn geen kosten verbonden aan het gebruik van deze tool.
%        De gebruiker betaalt enkel voor de onderliggende AWS-resources.
%        } \\\hline
%        AWS Elastic Beanstalk &
%            {\small
%        Er zijn geen kosten verbonden aan het gebruik van deze service.
%        De gebruiker betaalt enkel voor de onderliggende AWS-resources.
%        Het is aan de gebruiker om de grootte en het aantal AWS-resources te bepalen.
%        } \\\hline
%        \bottomrule
%    \end{tabular}
%    \caption{Vereenvoudiging tools: initiële inzet - kost}
%    \label{tab:inzet-kost-ver}
%\end{table}