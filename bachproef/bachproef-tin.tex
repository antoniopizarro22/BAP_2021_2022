%===============================================================================
% LaTeX sjabloon voor de bachelorproef toegepaste informatica aan HOGENT
% Meer info op https://github.com/HoGentTIN/bachproef-latex-sjabloon
%===============================================================================

\documentclass{bachproef-tin}

\usepackage{hogent-thesis-titlepage} % Titelpagina conform aan HOGENT huisstijl

%%---------- Documenteigenschappen ---------------------------------------------
% TODO: Vul dit aan met je eigen info:

% De titel van het rapport/bachelorproef
\title{Infrastructure as Code op AWS: analyse en vergelijkende studie van de beschikbare services}

% Je eigen naam
\author{Antonio Pizarro}

% De naam van je promotor (lector van de opleiding)
\promotor{Gertjan Bosteels}

% De naam van je co-promotor. Als je promotor ook je opdrachtgever is en je
% dus ook inhoudelijk begeleidt (en enkel dan!), mag je dit leeg laten.
\copromotor{Sander Adam}

% Indien je bachelorproef in opdracht van/in samenwerking met een bedrijf of
% externe organisatie geschreven is, geef je hier de naam. Zoniet laat je dit
% zoals het is.
\instelling{---}

% Academiejaar
\academiejaar{2021-2022}

% Examenperiode
%  - 1e semester = 1e examenperiode => 1
%  - 2e semester = 2e examenperiode => 2
%  - tweede zit  = 3e examenperiode => 3
\examenperiode{3}

%===============================================================================
% Inhoud document
%===============================================================================

\begin{document}

%---------- Taalselectie -------------------------------------------------------
% Als je je bachelorproef in het Engels schrijft, haal dan onderstaande regel
% uit commentaar. Let op: de tekst op de voorkaft blijft in het Nederlands, en
% dat is ook de bedoeling!

%\selectlanguage{english}

%---------- Titelblad ----------------------------------------------------------
\inserttitlepage

%---------- Samenvatting, voorwoord --------------------------------------------
\usechapterimagefalse
%%=============================================================================
%% Voorwoord
%%=============================================================================

\chapter*{\IfLanguageName{dutch}{Woord vooraf}{Preface}}
\label{ch:voorwoord}

TODO voorwoord

%% TODO:
%% Het voorwoord is het enige deel van de bachelorproef waar je vanuit je
%% eigen standpunt (``ik-vorm'') mag schrijven. Je kan hier bv. motiveren
%% waarom jij het onderwerp wil bespreken.
%% Vergeet ook niet te bedanken wie je geholpen/gesteund/... heeft


%%=============================================================================
%% Samenvatting
%%=============================================================================

% TODO: De "abstract" of samenvatting is een kernachtige (~ 1 blz. voor een
% thesis) synthese van het document.
%
% Deze aspecten moeten zeker aan bod komen:
% - Context: waarom is dit werk belangrijk?
% - Nood: waarom moest dit onderzocht worden?
% - Taak: wat heb je precies gedaan?
% - Object: wat staat in dit document geschreven?
% - Resultaat: wat was het resultaat?
% - Conclusie: wat is/zijn de belangrijkste conclusie(s)?
% - Perspectief: blijven er nog vragen open die in de toekomst nog kunnen
%    onderzocht worden? Wat is een mogelijk vervolg voor jouw onderzoek?
%
% LET OP! Een samenvatting is GEEN voorwoord!

%%---------- Nederlandse samenvatting -----------------------------------------
%
% TODO: Als je je bachelorproef in het Engels schrijft, moet je eerst een
% Nederlandse samenvatting invoegen. Haal daarvoor onderstaande code uit
% commentaar.
% Wie zijn bachelorproef in het Nederlands schrijft, kan dit negeren, de inhoud
% wordt niet in het document ingevoegd.

\IfLanguageName{english}{%
\selectlanguage{dutch}
\chapter*{Samenvatting}
\lipsum[1-4]
\selectlanguage{english}
}{}

%%---------- Samenvatting -----------------------------------------------------
% De samenvatting in de hoofdtaal van het document

\chapter*{\IfLanguageName{dutch}{Samenvatting}{Abstract}}

\lipsum[1-4]


%---------- Inhoudstafel -------------------------------------------------------
\pagestyle{empty} % Geen hoofding
\tableofcontents  % Voeg de inhoudstafel toe
\cleardoublepage  % Zorg dat volgende hoofstuk op een oneven pagina begint
\pagestyle{fancy} % Zet hoofding opnieuw aan

%---------- Lijst figuren, afkortingen, ... ------------------------------------

% Indien gewenst kan je hier een lijst van figuren/tabellen opgeven. Geef in
% dat geval je figuren/tabellen altijd een korte beschrijving:
%
%  \caption[korte beschrijving]{uitgebreide beschrijving}
%
% De korte beschrijving wordt gebruikt voor deze lijst, de uitgebreide staat bij
% de figuur of tabel zelf.

\listoffigures
\listoftables
\lstlistoflistings

% Als je een lijst van afkortingen of termen wil toevoegen, dan hoort die
% hier thuis. Gebruik bijvoorbeeld de ``glossaries'' package.
% https://www.overleaf.com/learn/latex/Glossaries

%---------- Kern ---------------------------------------------------------------

% De eerste hoofdstukken van een bachelorproef zijn meestal een inleiding op
% het onderwerp, literatuurstudie en verantwoording methodologie.
% Aarzel niet om een meer beschrijvende titel aan deze hoofstukken te geven of
% om bijvoorbeeld de inleiding en/of stand van zaken over meerdere hoofdstukken
% te verspreiden!

%%=============================================================================
%% Inleiding
%%=============================================================================

\chapter{\IfLanguageName{dutch}{Inleiding}{Introduction}}
\label{ch:inleiding}

De inleiding moet de lezer net genoeg informatie verschaffen om het onderwerp te begrijpen en in te zien waarom de onderzoeksvraag de moeite waard is om te onderzoeken. In de inleiding ga je literatuurverwijzingen beperken, zodat de tekst vlot leesbaar blijft. Je kan de inleiding verder onderverdelen in secties als dit de tekst verduidelijkt. Zaken die aan bod kunnen komen in de inleiding~\autocite{Pollefliet2011}:

\begin{itemize}
  \item context, achtergrond
  \item afbakenen van het onderwerp
  \item verantwoording van het onderwerp, methodologie
  \item probleemstelling
  \item onderzoeksdoelstelling
  \item onderzoeksvraag
  \item \ldots
\end{itemize}

\section{\IfLanguageName{dutch}{Probleemstelling}{Problem Statement}}
\label{sec:probleemstelling}

Uit je probleemstelling moet duidelijk zijn dat je onderzoek een meerwaarde heeft voor een concrete doelgroep. De doelgroep moet goed gedefinieerd en afgelijnd zijn. Doelgroepen als ``bedrijven,'' ``KMO's,'' systeembeheerders, enz.~zijn nog te vaag. Als je een lijstje kan maken van de personen/organisaties die een meerwaarde zullen vinden in deze bachelorproef (dit is eigenlijk je steekproefkader), dan is dat een indicatie dat de doelgroep goed gedefinieerd is. Dit kan een enkel bedrijf zijn of zelfs één persoon (je co-promotor/opdrachtgever).

\section{\IfLanguageName{dutch}{Onderzoeksvraag}{Research question}}
\label{sec:onderzoeksvraag}

Wees zo concreet mogelijk bij het formuleren van je onderzoeksvraag. Een onderzoeksvraag is trouwens iets waar nog niemand op dit moment een antwoord heeft (voor zover je kan nagaan). Het opzoeken van bestaande informatie (bv. ``welke tools bestaan er voor deze toepassing?'') is dus geen onderzoeksvraag. Je kan de onderzoeksvraag verder specifiëren in deelvragen. Bv.~als je onderzoek gaat over performantiemetingen, dan 

\section{\IfLanguageName{dutch}{Onderzoeksdoelstelling}{Research objective}}
\label{sec:onderzoeksdoelstelling}

Wat is het beoogde resultaat van je bachelorproef? Wat zijn de criteria voor succes? Beschrijf die zo concreet mogelijk. Gaat het bv. om een proof-of-concept, een prototype, een verslag met aanbevelingen, een vergelijkende studie, enz.

\section{\IfLanguageName{dutch}{Opzet van deze bachelorproef}{Structure of this bachelor thesis}}
\label{sec:opzet-bachelorproef}

% Het is gebruikelijk aan het einde van de inleiding een overzicht te
% geven van de opbouw van de rest van de tekst. Deze sectie bevat al een aanzet
% die je kan aanvullen/aanpassen in functie van je eigen tekst.

De rest van deze bachelorproef is als volgt opgebouwd:

In Hoofdstuk~\ref{ch:stand-van-zaken} wordt een overzicht gegeven van de stand van zaken binnen het onderzoeksdomein, op basis van een literatuurstudie.

In Hoofdstuk~\ref{ch:methodologie} wordt de methodologie toegelicht en worden de gebruikte onderzoekstechnieken besproken om een antwoord te kunnen formuleren op de onderzoeksvragen.

% TODO: Vul hier aan voor je eigen hoofstukken, één of twee zinnen per hoofdstuk

In Hoofdstuk~\ref{ch:conclusie}, tenslotte, wordt de conclusie gegeven en een antwoord geformuleerd op de onderzoeksvragen. Daarbij wordt ook een aanzet gegeven voor toekomstig onderzoek binnen dit domein.
% TODO: literatuurstudie + wat anderen denken
\chapter{\IfLanguageName{dutch}{Stand van zaken}{State of the art}}
\label{ch:stand-van-zaken}

\acrfull{iac} is een essentiële bouwsteen binnen de DevOps methodologie.
De infrastructuur code wordt behandeld op dezelfde manier als een softwareontwikkelaar zijn code zou behandelen.
Met andere woorden, de code wordt op een ordelijke manier binnen een versiebeheer systeem bewaard.
Daarnaast moet het ook mogelijk zijn om \acrfull{ci} en \acrfull{cd} toe te passen op de infrastructuur code.
De code moet een consistent en repliceerbaar resultaat leveren die automatisatie toelaat~\autocite{Mansoor2014}.

\section{AWS AppRunner}
\label{sec:service-apprunner}

AppRunner is een AWS-service die de functionaliteit verschaft om broncode of een container ~\gls{image} te inzetten op een snelle, simpele en kostefficiënte manier.
De toepassing wordt op een schaalbare en veilige AWS-cloud omgeving opgezet.
De gebruiker hoeft de configuratie van AWS-resources niet zelf in handen te nemen, alle achterliggende infrastructuur wordt door AppRunner beheerd~\autocite{Khen2022}.

In een artikel van \textcite{Aussems2021} beschrijft hij AWS AppRunner als een geautomatiseerde versie van AWS Fargate, de serverless containerdienst van de provider.
Daarom is deze service bestemd voor ontwikkelaars die geen tijd willen besteden in de uitrol, configuratie of beheer van hun containertoepassingen.
Ze hoeven enkel de broncode af te leveren, in de vorm van een code repository of een container ~\gls{image}~\autocite{Khen2022}.

Toepassingen die via AppRunner zijn ingezet, worden op container instanties geplaatst.
Deze instanties verbruiken computer- en geheugenbronnen, waarvoor de gebruiker moet betalen.
Het geheugen en vCPU worden bij het opzet van AppRunner bepaald en op basis daarvan wordt de kost per GB bepaald.
De kosten omvatten aan de ene kant de systeembronnen van een bevoorraadde container instantie.
Deze beperkte systeembronnen zijn nodig om de container instantie op een inactief staat te behouden.
Aan de andere kant moet er ook betaald worden voor de systeembronnen wanneer een container actief wordt gebruikt, bijvoorbeeld tijdens het verwerken van requests.
Er wordt enkel betaald als de toepassing draaiende is.
Het is heel eenvoudig om een AppRunner toepassing tijdelijk te stoppen en terug te starten.
Deze acties kunnen via de console, \acrshort{cli} of \acrshort{api} uitgevoerd worden~\autocite{AWSAppRunnerPricing}.

%marktbevindingen
\paragraph{Marktevaluatie}

%https://financesonline.com/news/building-saas-on-aws-now-faster-with-ga-release-of-low-code-tool-amplify-studio/
%https://bejamas.io/discovery/hosting/aws-amplify/
%Amplify focuses on user experience and gives frontend developers one of the easiest ways to use and connect various AWS products.

%TODO: Marktevaluatie
%In een AWS blog van \textcite{Spittel2022} legt hij uit hoe AWS Amplify Studio gemaakt is om ontwikkelaars

\section{AWS Amplify}
\label{sec:service-amplify}

AWS Amplify bestaat uit een set van tools gericht voor het ontwikkelen van frontend web en mobiele toepassingen.
Met deze tools wordt het opzetten van een full-stack toepassing op AWS-infrastructuur op een intuïtieve manier mogelijk~\autocite{AWSCopilotOverview}.
De gebruiker heeft de flexibiliteit om het grote aanbod aan AWS-componenten te integreren met zijn oplossing.
Integratie met backend componenten wordt mogelijk gemaakt via serverless technologieën~\autocite{Kandaswamy2022}.

Vervolgens worden de door Amplify aangeboden tools verder uitgelicht.

\paragraph{Amplify libraries}

De Amplify open-source client libraries voorzien use-case gebaseerde, declaratieve en gemakkelijk te gebruiken interfaces
doorheen verschillende categorieën van cloud ondersteunende operaties.
Deze stellen mobiele en web ontwikkelaars in staat om integraties met hun backend infrastructuur te realiseren.
De libraries worden ondersteund door de AWS-cloud en bieden een inplugbaar model aan, dat ook door andere cloud leveranciers kan worden gebruikt.
Verder kunnen de libraries gebruikt worden met beide, nieuwe backend aangemaakt via de Amplify \acrshort{cli} en bestaande backend bronnen~\autocite{AWSAmplifyDocs}.

\paragraph{Amplify Studio}

AWS Amplify Studio is een visuele ontwikkeling omgeving voor het bouwen van full-stack web en mobiele toepassingen.
Studio bouwt verder op bestaande backend-bouw mogelijkheden in AWS Simplify, UI-ontwikkeling wordt aanzienlijk versneld.
Met Studio is het mogelijk om een webtoepassing in zijn geheel te bouwen, front-to-back, met minimale codeer werk.
En toch behoudt de gebruiker volledige controle over de toepassing ontwerp en gedrag via code~\autocite{AWSAmplifyDocs}.

\paragraph{Amplify \acrshort{cli}}

%The Amplify Command Line Interface (CLI) is a unified toolchain to create, integrate, and manage the AWS cloud services for your app.
De Amplify \acrfull{cli} is een eendrachtig tool die het bouwen, integreren en beheren van AWS-services voor een toepassing faciliteert~\autocite{AWSAmplifyDocs}.
%TODO: verder uitleggen

\paragraph{Amplify Hosting}

%Amplify Hosting provides a git-based workflow for hosting full-stack serverless web apps with continuous deployment.
%This user guide provides the information you need to get started with Amplify Hosting.

Amplify Hosting biedt een git-gebaseerde workflow aan voor het hosten van een full-stack serverless webtoepassing met \acrfull{cd}).
%TODO: verder uitleggen

\paragraph{Marktevaluatie}

%https://financesonline.com/news/building-saas-on-aws-now-faster-with-ga-release-of-low-code-tool-amplify-studio/
%https://bejamas.io/discovery/hosting/aws-amplify/
%Amplify focuses on user experience and gives frontend developers one of the easiest ways to use and connect various AWS products.

In een AWS blog van \textcite{Spittel2022} legt hij uit hoe AWS Amplify Studio gemaakt is om het leven van ontwikkelaars gemakkelijker te maken.
De ontwerp-ontwikkelaar overdracht gebeurt vlotter d.m.v.\ Amplify.
Bovendien is de code aanpassen of uitbreiden heel eenvoudig met de gegenereerde componenten van Amplify Studio.

Ook \textcite{Nwamba2022} vertelt in zijn artikel hoe ontwikkelaars steeds meer low-code tools adopteren.
Ontwikkelaars die het ontwikkelwerk voor het bouwen van een toepassing interface willen vereenvoudigen moeten AWS Amplify overwegen.

\section{AWS Lightsail}
\label{subsec:service-lightsail}

AWS Lightsail verschaft cloud systeembronnen om een toepassing of webtoepassing moeiteloos te kunnen opzetten.
Lightsail biedt vereenvoudigde diensten aan zoals instanties, containers, databanken, opslag, enz.
Niet alleen is Lightsail compatibel met voorgeconfigureerde blueprints zoals Wordpress, Prestashop of LAMP\@.
Maar het is ook mogelijk om statische content te hosten met Lightsail en die globaal ter beschikking te stellen~\autocite{AWSLightsail2022}.

Verder worden een aantal kenmerken van Lightsail uitgelicht.

\begin{description}
    \item[Instances: ] Lightsail virtuele servers voor het hosten van toepassingen.
    \item[Containers: ] Inzetten van Docker containers via Lightsail.
    \item[Simplified load balancers: ] Vereenvoudigde load balancing functionaliteit.
    \item[Managed databases: ] Volledig geconfigureerde MySQL of PostgreSQL databanken.
    \item[Block and object storage: ] Schaalbaar block en object opslag op de AWS-cloud.
    \item[CDN distributions: ] \acrfull{cdn} voor de globale distributie van toepassingen.
\end{description}

\paragraph{Marktevaluatie}

In een blog van \textcite{Warrier2022} beschrijft hij Lightsail als een gemakkelijke tool om eenvoudige workloads te beheren
en voorgeconfigureerde infrastructuur, zoals toepassingen, databanken en andere populaire stacks, te provisioneren.

\section{AWS Copilot}
\label{subsec:service-copilot}

AWS Copilot is een \acrfull{cli} voor ontwikkelaars die gecontaineriseerde toepassingen willen inzetten op AWS-services zoals AWS AppRunner, AWS Elastic container service (ECS) en AWS Fargate.
De volledige levenscyclus van de toepassing kan via Copilot worden beheerd.
Componenten zoals omgevingen en services kunnen m.b.v.\ de tool worden geconfigureerd.
Vervolgens worden de hoofdconcepten van Copilot uitgelicht~\autocite{Karakus2022}.

\paragraph{Applications}

Een applicatie is een collectie van services en omgevingen.
De applicatienaam is dan meestal een high-level beschrijving van het product.
Bijvoorbeeld een applicatie met naam "Chat" kan twee services bevatten, "frontend" en "api".
Verder kunnen de services ingezet worden op een "test" en "production" omgeving~\autocite{Karakus2022}.

\paragraph{Environments}

Het concept van omgevingen is door iedere ontwikkelaar gekend.
Voordat de toepassing op een productieve omgeving kan worden vrijgesteld is het aangeraden om eerst een test omgeving te voorzien.
Deze testomgeving is dan een weerspiegeling van wat er uiteindelijk in productie zal staan.

Elke omgeving kan zijn eigen versie van een service bevatten.
De optimale werkwijze zou zijn om, voordat een service ingezet wordt op de productieomgeving, die eerst te testen op de testomgeving.

Verder bestaat ook de mogelijkheid om een nieuwe service aan een bestaande omgeving toe te voegen.
Elke omgeving bevat een set van AWS-resources die gedeeld wordt binnen die omgeving.
Deze omvatten netwerk-resources, de \acrshort{ecs}-cluster en de load-balancer.
Een nieuwe service zou gebruik maken van deze gedeelde resources~\autocite{Karakus2022}.

\paragraph{Services}

Een service bestaat uit de code van de toepassing en alle ondersteunende infrastructuur die nodig is om de toepassing op AWS te kunnen hosten.
De gebruiker kiest via de \acrshort{cli} welke type service moet worden aangemaakt.
Afhankelijk van het type service worden de nodige AWS-resources bevoorraad.
Ter illustratie, een gebruiker kan kiezen voor een service om trafiek van het internet te bedienen.
Daarbij horen AWS-resources zoals een \acrfull{alb} en een \acrshort{ecs} service op AWS Fargate.

Als de gebruiker een service type heeft gekozen, wordt er een Dockerfile gegenereerd en de container \glspl{image} worden bewaard op \acrfull{ecr}.
Ook wordt er een \emph{manifest} bestand aangemaakt, deze bevat configuratie parameters voor de service zoals hoeveelheid geheugen, CPU, het aantal kopieën van de service en andere~\autocite{Karakus2022}.

\paragraph{Jobs}

Jobs zijn vluchtig ECS tasks die getriggerd worden door een event.
Wanneer een job voldaan is, wordt de \acrshort{ecs} task beëindigd.
De gebruiker dient alle nodige configuratie te voorzien voor de opzet van de \acrshort{ecs} task.
Daarnaast wordt er ook een \emph{manifest} bestand voorzien waar geavanceerde configuratie kan worden ingesteld~\autocite{Karakus2022}.

\paragraph{Pipelines}

Een \acrfull{cd} pipeline kan worden opgezet via Copilot.
Copilot is compatibel met de volgende git repositories: GitHub, BitBucket en CodeCommit.
Wanneer een push wordt gedetecteerd, de pipeline bouwt de service, bewaart de container ~\gls{image} op ~\acrshort{ecr} en begint de inzet naar een omgeving~\autocite{Karakus2022}.

\section{AWS Elastic Beanstalk}
\label{subsec:service-elastic-beanstalk}

~\acrfull{eb} is een andere \acrshort{aws} service die het inzetten van toepassingen op AWS-infrastructuur aanzienlijk vereenvoudigt.
Het beheer van AWS-infrastructuur wordt grotendeels door \acrshort{eb} uitgevoerd.
En toch behoudt de gebruiker de mogelijkheid om configuratiekeuzen te maken.

~\acrlong{eb} ondersteunt toepassingen ontwikkeld in Go, Java, .NET, Node.js, PHP, Python en Ruby.
Wanneer een toepassing wordt ingezet, bouwt \acrshort{eb} de geselecteerd platform versie en voorziet de nodige AWS-resources.
De gebruiker kan \acrlong{eb} gebruiken via de AWS \acrfull{cli}, of eb, een high-level \acrshort{cli} doelbewust voor \acrlong{eb} ontworpen.

Bovendien is het ook mogelijk om via \acrshort{eb} alledaagse inzet gerelateerde taken uit te voeren.
Zoals de grootte van de \acrshort{ec2} fleet instanties aanpassen of het monitoren van de toepassingen.
Deze acties kunnen rechtstreeks vanuit het \acrshort{eb} webinterface (console) uitgevoerd worden~\autocite{Khen2022a}.

Het gebruik van \acrlong{eb} bedraagt geen bijkomende kosten.
De gebruiker betaalt voor de AWS-resources die ingezet worden om de toepassing op AWS te hosten~\autocite{Khen2022a}.

\section{AWS Cloudformation}
\label{subsec:service-cloudformation}

Cloudformation is een AWS-service voor het modelleren en opzetten van AWS-resources.
Complexe opstellingen, waarbij meerdere AWS-resources van elkaar afhankelijk zijn en samen functioneren, kunnen als één geheel geconfigureerd worden.
Een Cloudformation template moet meerdere keren kunnen gebruikt worden.
Het is daarom ook mogelijk om via parameters de template dynamisch te maken.
De gebruiker kan a.d.h.v.\ deze parameters de configuratie van de AWS-resources anders instellen zonder de template elke keer te moeten aanpassen.
Voor infrastructuur en AWS-resources die via Cloudformation zijn opgezet bestaat de mogelijkheid om aanpassingen te brengen via change-sets.
Een alternatief is om bestaande infrastructuurcomponenten in een Cloudformation stack te opnemen.
Daarnaast biedt Cloudformation ook de functionaliteit aan om infrastructuur eenvoudig af te bouwen~\autocite{Mansoor2014}.

\paragraph{Template}

Cloudformation werkt op basis van templates.
Een template is een JSON- of YAML-bestand waarin de gewenste AWS-resources en configuratie worden beschreven.
Deze templates moeten op een \acrshort{s3} bucket beschikbaar gesteld worden om door Cloudformation gebruikt te kunnen worden.
De mogelijkheid bestaat ook om naar een bestand op een lokaal systeem te verwijzen via de console, \acrshort{api} of \acrshort{cli}.
AWS zorgt in dat geval dat het bestand wordt opgeladen op een \acrshort{s3} bucket.
Templates kunnen ook dynamisch geïmplementeerd worden door gebruik te maken van parameters.
Deze parameters worden gespecifieerd bij het aanmaken of actualiseren van een stack.
Zo wordt er vermeden om telkens opnieuw de template te moeten aanpassen~\autocite{AWSCLoudformationUser}.

\paragraph{Stack}

Een Cloudformation stack is een collectie van AWS-resources die beheerd kunnen worden als één geheel.
De stack wordt aangemaakt op basis van een Cloudformation template.
Indien de template parameters bevat, dan kunnen die gespecifieerd worden bij het aanmaken van de stack.
Anders bestaat ook de mogelijkheid om default waarden in te stellen voor de parameters.
Nadat een stack is aangemaakt kan die geactualiseerd worden a.d.h.v.\ change-sets.
De aangemaakte resources kunnen eenvoudig afgebouwd worden door de stack te verwijderen~\autocite{AWSCLoudformationUser}.

\paragraph{Change set}

De resources van een bestaande stack kunnen aangepast worden via een change set.
Daarvoor moet één of beide van de volgende acties uitgevoerd worden:

\begin{itemize}
    \item Aanpassingen brengen aan de bestaande template.
    \item Nieuwe input parameters specifiëren.
\end{itemize}

Een change set is de vergelijking van de huidige template t.o.v.\ de aangepaste versie.
In de change set worden de voorgestelde aanpassingen opgesomd.
Eens goedgekeurd worden de resources voor die stack bijgewerkt~\autocite{AWSCLoudformationUser}.

\paragraph{Delete stack}

Bij het verwijderen van een stack worden de betreffende AWS-resources afgebouwd.
Het is mogelijk om bepaalde resources niet te verwijderen door een “deletion policy” te configureren waarbij de gewenste resources behouden blijven.
Indien één van de resources niet verwijderd kan worden dan kan de stack ook niet verwijderd worden.
In dat geval blijft de stack bestaan tot de systeemcomponenten succesvol verwijderd kunnen worden, of als alternatief kan aangegeven worden om deze te behouden buiten de stack~\autocite{AWSCLoudformationUser}.

\section{Terraform}
\label{subsec:service-terraform}

Terraform is een HashiCorp open-source ~\acrfull{iac} tool.
Hashicorp is een Advanced Technology Partner bij ~\acrfull{apn} en ook een lid van de AWS DevOps Competency ~\autocite{Campbell2018}.
Terraform is een gelijkaardige tool aan Cloudformation.
Beide werken met declaratieve configuratie bestanden waarmee AWS-infrastructuur kan worden ingezet en geactualiseerd.

Terraform is platform-agnostisch;
de tool is compatibel met on-site servers of cloud servers zoals AWS, Google Cloud Platform, OpenStack en Azure.
Het is mogelijk om meerdere cloud-leveranciers, voor verschillende doeleinden, te gebruiken.
Dit betekent niet dat de gebruiker zonder gevolgen van cloud-leverancier kan schakelen.
De cloud-providers hebben elk een andere werkmethodologie en aanbod aan resources.
Daarom is spontaan van cloud-provider nooit een optimale oplossing.
Toch is de flexibiliteit en compatibiliteit van Terraform een grote troef~\autocite{Szalski2019}.

Terraform werkt met configuratie bestanden zoals de meeste declaratieve \acrshort{iac} tools.
Deze bestanden worden geschreven in ~\acrfull{hcl}, de configuratie taal van HashiCorp.
De syntax heeft eigenschappen zoals variabelen en interpolatie, en de documentatie is zeer goed.
In een Terraform configuratie bestand worden resources en data sources beschreven.
Een resource vertegenwoordigt een component van de infrastructuur zoals een AWS ~\acrshort{ec2} instantie, een \acrshort{rds} instantie, een Route53 ~\acrshort{dns} record of een regel in een ~\gls{security-group}.
Op deze manier kan de gebruiker de resources configureren en inzetten op de cloud architectuur~\autocite{Szalski2019}.

\section{Ansible}
\label{subsec:service-ansible}

%TODO: literature Ansible

\section{AWS SAM}
\label{subsec:service-sam}

%TODO: literature SAM

\section{AWS CDK}
\label{subsec:service-cdk}

\acrfull{cdk} is een open source software development framework gebruikt voor het schrijven van \acrfull{iac} in een voor de ontwikkelaar vertrouwde programmeertaal zoals:

\begin{itemize}
    \item TypeScript
    \item JavaScript
    \item Python
    \item Java
    \item C\#
\end{itemize}

AWS \acrshort{cdk} gebruikt de voordelen van een programmeertaal om infrastructuur en toepassingen te schrijven, lezen en begrijpen~\autocite{Mansoor2014}.

De \acrshort{cdk}-framework compileert de code en genereert een Cloudformation stack.
Daarom gelden dezelfde principes op deze tool als bij Cloudformation.
Via de \acrshort{cdk}-Toolkit, een \acrshort{cli}-interface voor \acrshort{cdk}, kunnen de standaard acties zoals een stack creëren, wijzigen en verwijderen uitgevoerd worden.
Daarnaast zijn troubleshooting functionaliteiten ook beschikbaar via de \acrshort{cdk}-Toolkit~\autocite{Mansoor2014}.

\paragraph{App}

De App construct is de root klasse van alle constructs.
De klasse heeft geen argumenten nodig.
Bij het aanmaken van een Stack construct moet er verwezen naar een App construct.

\paragraph{Stack}

Een stack is de eenheid van inzet voor AWS \acrshort{cdk}.
Een \acrshort{cdk}-stack is gelijkwaardig aan een Cloudformation-stack en beschikt over dezelfde beperkingen.
Alle AWS-resources die gedefinieerd worden binnen een stack worden als één geheel beschouwd.

\paragraph{Construct}

Een construct ligt aan de basis van een \acrshort{cdk}-app.
Het staat voor een AWS-resource en bevat alle configuratie die nodig is voor Cloudformation om de component te bouwen.
Een construct kan één AWS-resource vertegenwoordigen zoals een AWS S3 bucket of kan bestaan uit meerdere resources waarbij men spreekt over een higher-level abstraction construct.
De AWS \acrshort{cdk} library bevat een collectie van constructs voor elke AWS-resource.
Er bestaan ook meer complexe third-party constructs die beschikbaar gesteld worden via de Construct Hub.

\paragraph{CDK-Toolkit}

Met behulp van de toolkit \acrshort{cli} kunnen alle functionaliteiten uitgevoerd worden die nodig zijn om de applicatie en infrastructuur te kunnen inzetten en beheren zoals:

\begin{description}
    \item[Synthesis:] De synth commando compileert de code en genereert een Cloudformation template.
    \item[Deploy:] De deploy commando zal de stack inzetten naar de geconfigureerde AWS-account, voor elke inzet wordt de code gesynthetiseerd.
    \item[Diff:] De diff commando vergelijkt de laatste stack met de laatste gesynthetiseerde template, de output is een overzicht van de verschillen tussen de twee.
    \item[Destroy:] De destroy commando verwijdert de stack en de bijhorende AWS-resources, dit is gelijkwaardig aan het verwijderen van een Cloudformation stack.
\end{description}

\section{Troposphere}
\label{sec:service-troposphere}

%TODO: literature Troposphere

\section{Pulumi}
\label{sec:service-pulumi}

%TODO: literature Pulumi % console-based services/tools eerst uitschrijven -
%%=============================================================================
%% Methodologie
%%=============================================================================

\chapter{\IfLanguageName{dutch}{Methodologie}{Methodology}}
\label{ch:methodologie}

%% TODO: Hoe ben je te werk gegaan? Verdeel je onderzoek in grote fasen, en
%% licht in elke fase toe welke stappen je gevolgd hebt. Verantwoord waarom je
%% op deze manier te werk gegaan bent. Je moet kunnen aantonen dat je de best
%% mogelijke manier toegepast hebt om een antwoord te vinden op de
%% onderzoeksvraag.

%TODO: verwijzen naar bronnen of zelf bepalen? voor bepaalde criteria eigen waardering
%TODO: basiskennis voor de AWS-infra wordt in rekening gebracht

Dit onderzoek bestaat uit een literatuurstudie waarbij de verschillende~\acrfull{iac} services in kaart worden gebracht.
In dit onderzoek wordt specifiek gekeken naar tools en services die compatibel zijn met~\acrshort{aws}.
Het is dan ook geen verrassing dat de meeste services door~\acrshort{aws} zelf worden aangeboden.
De literatuurstudie is te vinden in Hoofdstuk~\ref{sec:criteria-evaluatie}.

In Hoofdstuk~\ref{ch:vergelijking} wordt de evaluatie en vergelijking gedaan van de eerder besproken~\arcshort{iac} tools.
De criteria, als beschreven in Hoofdstuk~\ref{sec:criteria-evaluatie}, wordt toegepast op de volgende tools:

\begin{itemize}
    \item AWS AppRunner
    \item AWS Amplify
    \item AWS Lightsail
    \item AWS Copilot
    \item AWS Elastic Beanstalk
    \item AWS Cloudformation
    \item Terraform
    \item Ansible
    \item AWS SAM
    \item AWS CDK
    \item Troposphere
    \item Pulumi
\end{itemize}

Om op de criteria te kunnen antwoorden wordt er gekeken naar verschillende bronnen:

\begin{itemize}
    \item Documentatie tools/services
    \item Artikels geschreven door gebruikers
    \item Startersgidsen
\end{itemize}

In Hoofdstuk~\ref{ch:business-case} wordt een business case uitgewerkt.
De business case zal bestaan uit een aantal requirements waaraan de \acrshort{iac} tool moet voldoen.
De tools die de business case werkelijk kunnen uitwerken, zonder uitbreidingen naar andere services, zullen geselecteerd worden.
In Hoofdstuk~\ref{ch:uitwerking} wordt de business case uitgewerkt met de geselecteerde tools.

% beschrijving criteria + literatuurstudie -> vergelijking ->  op basis van business case + criteria -> uitwerking

\section{Criteria evaluatie}
\label{sec:criteria-evaluatie}

De services en tools zullen op bepaalde criteria geëvalueerd worden.
De criteria is grotendeels afhankelijk van het type onderneming, een startup, een KBO of een grote onderneming.
Aan de hand van een vergelijkende studie wordt er bepaald welke tool in welke situatie het best wordt gebruikt.

\begin{enumerate}
    \item Initiële inzet
    \begin{itemize}
        \item Training
        \item Kost opzet
        \item Documentatie en/of support
    \end{itemize}
    \item Operationele effectiviteit
    \begin{itemize}
        \item State – Validatie – Rollback – Drift elimination % TODO: (Procesfunctionaliteiten) beter naam - vanaf rollback +
        \item Maturiteit
        \item Onderhoudbaarheid
        \item Veiligheid (least privilege implementation, RBAC, transparantie)
        \item Categorisatie (tagging, resource groups, omgevingen)
    \end{itemize}
    \item Gebruiksgemak/Gebruikscomfort
    \begin{itemize}
        \item Herbruikbaarheid
        \item Ready-made (abstractieniveau)
        \item Cross-platform support
        \item Syntax
        \item Open-source
    \end{itemize}
    \item \acrfull{tco}
    \begin{itemize}
        \item Infrastructuur
        \item Support
    \end{itemize}
    \item Flexibiliteit
    \begin{itemize}
        \item Integratie met een bestaande omgeving
        \item Integratie/collaboratie met andere IaC tools
        \item Integratie met andere DevOps tools
        \item Integratie met andere development frameworks
    \end{itemize}
\end{enumerate}

In de eerste plaats wordt er gekeken naar de eerste inzet die nodig is om de tool of service in gebruik te nemen.
Dit houdt in de kosten gebonden aan het gebruik van de tool, eventuele trainingen als het gaat over een complexe tools die voorkennis vereisen.
De documentatie is ook een grote aanwijzing voor de waarde van de tool.
Goed gestructureerde en gesubstantieerde documentatie maakt een eerste kennismaking met de tool moeiteloos.

De operationele effectiviteit van de tool wordt ook onder de loep genomen.
De maturiteit en de onderhoudbaarheid van de tool spelen een grote rol als een gebruiker een keuze moet maken tussen de verschilleden tools.
Een tool met een lange levensduur die nog steeds wordt onderhouden zal eerder als betrouwbaar geacht worden.
Binnen IaC zijn er bepaalde functionaliteiten die de gebruiker op een georganiseerde manier laten werken en het beheer van de infrastructuur ondersteunen.
Het al dan niet beschikken van deze functionaliteiten zal ook wegen bij het evalueren van de tools.
Een aantal van deze functionaliteiten wordt hieronder beschreven.

\begin{description}
    \item[State:] De tool geeft de mogelijkheid om de status van de gereproduceerde infrastructuur op te volgen. Als er wijzigingen zouden plaatsvinden dan worden deze bij de state opgevangen.
    \item[Validatie:] Als er sprake is van code dan wordt er verwacht dat deze aan bepaalde syntax of formaat eisen voldoet. De validatie moet ervoor zorgen dat voor het genereren van de infrastructuur de nodige meldingen worden weergegeven als iets niet klopt.
    \item[Rollback:] De mogelijkheid om de gegenereerd infrastructuur te verwijderen of wijzigingen terug te draaien.
    \item[Drift elimination:] De functionaliteit om afwijkingen tussen de state en de huidige implementatie te kunnen detecteren. Wijzigingen die buiten de scope van de IaC tool worden aangebracht zijn in dit geval een afwijking.
    \item[Veiligheid:] Op vlak van veiligheid wordt er gekeken naar de functionaliteit rond toegankelijkheid. Het is van uiterst belang dat enkel toegestane gebruikers infrastructuur kunnen genereren of wijzigen. Concepten zoal least privilege, roll based access control (RBAC) en policies horen hierbij.
    \item[Categorisatie:] Functionaliteit om infrastructuur te beheren op een ordelijke manier zoals tagging, resource groups en omgevingen.
\end{description}

Onder de categorie gebruiksgemak worden zaken zoals de herbruikbaarheid van de tool bestudeerd.
De mogelijkheid om verschillende omgevingen te kunnen opzetten vertrekkende van dezelfde infrastructuur template.
De complexiteit van de tool is een determinerende factor.
Sommige tools eisen de volledige configuratie van de gedefinieerde AWS-resources.
Andere tools vragen helemaal geen configuratie, de gebruiker moet zich dan enkel bezighouden met de ontwikkeling van de toepassing.
Afhankelijk van het niveau van abstractie kan een tool heel complex of heel eenvoudig zijn.
De syntax is een factor die meespeelt bij tools waar het vertrekkende punt een template of code is.
Als de code in een programmeertaal, die niet exclusief is voor die tool, is geschreven dan moet de code voldoen aan de regels van dat programmeertaal.
Bepaalde tools beschikken over hun eigen syntax, andere gebruiken generieke bestandsformaten zoals JSON en YML.

De~\acrfull{tco} is een andere belangrijke factor.
Het budget waarover een onderneming beschikt zal ook meespelen bij de keuze van een IaC tool.
Niet alle ondernemingen kunnen de kosten om een complexe tool te gebruiken verantwoorden, de expertise die nodig is om de tool in gebruik te nemen kan extra kosten betekenen.

% TODO: herschrijven
%\section{Services en tools}
%\label{sec:services-tools}
%
%Er is een breed aanbod aan services en tools voor het opzetten van AWS-resources.
%De tools en services die verder besproken zullen worden variëren van interne AWS-services tot derde partij tools.
%De services en tools verschillen ook op vlak van complexiteit.
%Er zijn fully-managed services die de configuratie van AWS-resources grotendeels of deels op zich nemen, de gebruiker
%kan eenvoudig toepassingen inzetten en beheren zonder veel te moeten nadenken over de achterliggende resources.

 % beschrijving criteria + literatuurstudie -> vergelijking ->  op basis van business case + criteria -> uitwerking
%%=============================================================================
%% Vergelijking
%%=============================================================================

\chapter{\IfLanguageName{dutch}{Vergelijking services en tools}{Comparison services and tools}}
\label{ch:vergelijking}

TODO % alle services/tools + magic quadrant
%%=============================================================================
%% Business case
%%=============================================================================

\chapter{\IfLanguageName{dutch}{Business case}{Business case}}
\label{ch:business-case}

\section{Functionele requirements}
\label{sec:functionele-requirements}

\begin{itemize}
    \item Een opslagplaats waar bestanden langdurig en veilig kunnen bewaard worden.
    \item De metadata van de bewaarde bestanden wordt automatisch opgeslagen.
    \item Een API-service om de metadata van de bewaarde bestanden te kunnen raadplegen en/of te bewerken.
\end{itemize}

\section{AWS Serverless oplossing}
\label{sec:serverless-oplossing}

De serverless oplossing bestaan uit de volgende AWS-resources:

\begin{description}
    \item[AWS S3 bucket] opslag voor de bestanden
    \item[AWS Lambda functie 1] S3 events verwerker - bij het toevoegen van een bestand aan de S3 bucket
    \item[AWS Lambda functie 2] Backend voor de API-gateway, verwerker van de API-calls
    \item[AWS API-gateway] API-service voor het raadplegen of bewerken van de metadata
    \item[AWS DynamoDB] NoSQL databank voor het opslaan van de metadata
\end{description}

 % beschrijving ondenming + business case -- oplossing komt hier niet
%%=============================================================================
%% Verwerking
%%=============================================================================

\chapter{\IfLanguageName{dutch}{Uitwerking business case}{Use case}}
\label{ch:uitwerking}

%TODO: business case

TODO uitwerking cloudformation / AWS CDK / Terraform

\section{Uitwerking AWS CDK}
\label{sec:uitwerking-cdk}

TODO

\section{Uitwerking AWS Cloudformation}
\label{sec:uitwerking-cloudformation}

TODO

\section{Uitwerking Terraform}
\label{sec:uitwerking-terraform}

TODO % selectie + oplossing Cloudformation, AWS CDK

% Voeg hier je eigen hoofdstukken toe die de ``corpus'' van je bachelorproef
% vormen. De structuur en titels hangen af van je eigen onderzoek. Je kan bv.
% elke fase in je onderzoek in een apart hoofdstuk bespreken.

%\input{...}
%\input{...}
%...

%%=============================================================================
%% Conclusie
%%=============================================================================

\chapter{Conclusie}
\label{ch:conclusie}

% TODO: Trek een duidelijke conclusie, in de vorm van een antwoord op de
% onderzoeksvra(a)g(en). Wat was jouw bijdrage aan het onderzoeksdomein en
% hoe biedt dit meerwaarde aan het vakgebied/doelgroep? 
% Reflecteer kritisch over het resultaat. In Engelse teksten wordt deze sectie
% ``Discussion'' genoemd. Had je deze uitkomst verwacht? Zijn er zaken die nog
% niet duidelijk zijn?
% Heeft het onderzoek geleid tot nieuwe vragen die uitnodigen tot verder 
%onderzoek?

\lipsum[76-80]



%%=============================================================================
%% Bijlagen
%%=============================================================================

\appendix
\renewcommand{\chaptername}{Appendix}

%%---------- Onderzoeksvoorstel -----------------------------------------------

\chapter{Onderzoeksvoorstel}

Het onderwerp van deze bachelorproef is gebaseerd op een onderzoeksvoorstel dat vooraf werd beoordeeld door de promotor. Dat voorstel is opgenomen in deze bijlage.

% Verwijzing naar het bestand met de inhoud van het onderzoeksvoorstel
%---------- Inleiding ---------------------------------------------------------

\section{Introductie} % The \section*{} command stops section numbering
\label{sec:introductie}

Hier introduceer je werk. Je hoeft hier nog niet te technisch te gaan.

Je beschrijft zeker:

\begin{itemize}
  \item de probleemstelling en context
  \item de motivatie en relevantie voor het onderzoek
  \item de doelstelling en onderzoeksvraag/-vragen
\end{itemize}

%---------- Stand van zaken ---------------------------------------------------

\section{State-of-the-art}
\label{sec:state-of-the-art}

Hier beschrijf je de \emph{state-of-the-art} rondom je gekozen onderzoeksdomein. Dit kan bijvoorbeeld een literatuurstudie zijn. Je mag de titel van deze sectie ook aanpassen (literatuurstudie, stand van zaken, enz.). Zijn er al gelijkaardige onderzoeken gevoerd? Wat concluderen ze? Wat is het verschil met jouw onderzoek? Wat is de relevantie met jouw onderzoek?

Verwijs bij elke introductie van een term of bewering over het domein naar de vakliteratuur, bijvoorbeeld~\autocite{Doll1954}! Denk zeker goed na welke werken je refereert en waarom.

% Voor literatuurverwijzingen zijn er twee belangrijke commando's:
% \autocite{KEY} => (Auteur, jaartal) Gebruik dit als de naam van de auteur
%   geen onderdeel is van de zin.
% \textcite{KEY} => Auteur (jaartal)  Gebruik dit als de auteursnaam wel een
%   functie heeft in de zin (bv. ``Uit onderzoek door Doll & Hill (1954) bleek
%   ...'')

Je mag gerust gebruik maken van subsecties in dit onderdeel.

%---------- Methodologie ------------------------------------------------------
\section{Methodologie}
\label{sec:methodologie}

Hier beschrijf je hoe je van plan bent het onderzoek te voeren. Welke onderzoekstechniek ga je toepassen om elk van je onderzoeksvragen te beantwoorden? Gebruik je hiervoor experimenten, vragenlijsten, simulaties? Je beschrijft ook al welke tools je denkt hiervoor te gebruiken of te ontwikkelen.

%---------- Verwachte resultaten ----------------------------------------------
\section{Verwachte resultaten}
\label{sec:verwachte_resultaten}

Hier beschrijf je welke resultaten je verwacht. Als je metingen en simulaties uitvoert, kan je hier al mock-ups maken van de grafieken samen met de verwachte conclusies. Benoem zeker al je assen en de stukken van de grafiek die je gaat gebruiken. Dit zorgt ervoor dat je concreet weet hoe je je data gaat moeten structureren.

%---------- Verwachte conclusies ----------------------------------------------
\section{Verwachte conclusies}
\label{sec:verwachte_conclusies}

Hier beschrijf je wat je verwacht uit je onderzoek, met de motivatie waarom. Het is \textbf{niet} erg indien uit je onderzoek andere resultaten en conclusies vloeien dan dat je hier beschrijft: het is dan juist interessant om te onderzoeken waarom jouw hypothesen niet overeenkomen met de resultaten.



%%---------- Andere bijlagen --------------------------------------------------
% TODO: Voeg hier eventuele andere bijlagen toe
%\input{...}
%TODO: woordenlijst + afkortingen + acroniemen
%%! Author = antonio
%! Date = 05/06/2022

% Preamble
%\documentclass[11pt]{article}

% Packages

% woordenlijst

\newglossaryentry{image}
{
    name=image,
    description={Een snapshot van een container waar er al bepaalde software is geïnstalleerd en/of geconfigureerd.}
}

\newglossaryentry{test}
{
    name=test1,
    description={Mathematics is what mathematicians do}
}

% acroniemen

\newacronym{aws}{AWS}{Amazon Web Services}
\newacronym{cd}{CD}{Continuous Deployment}
\newacronym{ci}{CI}{Continuous Integration}
\newacronym{iac}{IaC}{Infrastructure as Code}
\newacronym{cli}{CLI}{Command Line Interface}
\newacronym{api}{API}{Application programming interface}
\newacronym{cdn}{CDN}{Content delivery network}
\newacronym{ecs}{ECS}{Elastic Container Service}
\newacronym{alb}{ALB}{Application Load Balancer}
\newacronym{ecr}{ECR}{Elastic Container Repository}
\newacronym{s3}{S3}{Simple Storage Service}
\newacronym{ec2}{EC2}{Elastic Compute Cloud}
\newacronym{cdk}{CDK}{Cloud Development Kit}
\newacronym{eb}{EB}{Elastic Beanstalk}
%%---------- Referentielijst --------------------------------------------------

\printbibliography[heading=bibintoc]

\end{document}
