\chapter{\IfLanguageName{dutch}{Stand van zaken}{State of the art}}
\label{ch:stand-van-zaken}

Infrastructure as Code (IaC) is een essentiële bouwsteen binnen de DevOps methodologie.
De infrastructuur code wordt behandeld op dezelfde manier als een softwareontwikkelaar zijn code zou behandelen.
Met andere woorden, de code wordt op een ordelijke manier binnen een versiebeheer systeem bewaard.
Daarnaast moet het ook mogelijk zijn om continuous integration (CI) en continuous development (CD) toe te passen op de infrastructuur code.
De code moet een consistent en repliceerbaar resultaat leveren die automatisatie toelaat~\autocite{Mansoor2014}.

\subsection{AWS AppRunner}
\label{sec:service-apprunner}

AppRunner is een AWS-service die de functionaliteit verschaft om broncode of een container image te deployen op een snelle, simpele en kostefficiënte manier.
De toepassing wordt op een schaalbare en veilige AWS-cloud omgeving opgezet.
De gebruiker hoeft de configuratie van AWS-resources niet zelf in handen te nemen, alle achterliggende infrastructuur wordt door AppRunner beheerd~\autocite{Khen2022}.

In een artikel van \textcite{Aussems2021} beschrijft hij AWS AppRunner als een geautomatiseerde versie van AWS Fargate, de serverless containerdienst van de provider.
Daarom is deze service bestemd voor ontwikkelaars die geen tijd willen besteden in de uitrol, configuratie of beheer van hun containertoepassingen.
Ze hoeven enkel de broncode af te leveren, in de vorm van een code repository of een container image~\autocite{Khen2022}.

Toepassingen die via AppRunner zijn gedeployed worden op container instanties geplaatst.
Deze instanties verbruiken computer- en geheugenbronnen, waarvoor de gebruiker moet betalen.
Het geheugen en vCPU worden bij het opzet van AppRunner bepaald en op basis daarvan wordt de kost per GB bepaald.
De kosten omvatten aan de ene kant de systeembronnen van een bevoorraadde container instantie.
Deze beperkte systeembronnen zijn nodig om de container instantie op een inactief staat te behouden.
Daarnaast moet er ook betaald worden voor de systeembronnen wanneer een container actief wordt gebruikt, bijvoorbeeld tijdens het verwerken van requests.
Er wordt enkel betaald als de toepassing draaiende is.
Het is heel eenvoudig om een AppRunner toepassing tijdelijk te stoppen en terug te starten.
Deze acties kunnen via de console, CLI of API uitgevoerd worden~\autocite{AWSAppRunnerPricing}.

%marktbevindingen
\subsubsection{Marktevaluatie}

%https://financesonline.com/news/building-saas-on-aws-now-faster-with-ga-release-of-low-code-tool-amplify-studio/
%https://bejamas.io/discovery/hosting/aws-amplify/
%Amplify focuses on user experience and gives frontend developers one of the easiest ways to use and connect various AWS products.

%TODO: Marktevaluatie
%In een AWS blog van \textcite{Spittel2022} legt hij uit hoe AWS Amplify Studio gemaakt is om ontwikkelaars

\subsection{AWS Amplify}
\label{sec:service-amplify}

AWS Amplify bestaat uit een set van tools gericht voor het ontwikkelen van frontend web en mobiele toepassingen.
Met deze tools wordt het opzetten van een full-stack toepassing op AWS-infrastructuur op een intuïtieve manier mogelijk~\autocite{AWSCopilotOverview}.
De gebruiker heeft de flexibiliteit om het grote aanbod aan AWS-componenten te integreren met zijn oplossing.
Integratie met backend componenten wordt mogelijk gemaakt via serverless technologieën~\autocite{Kandaswamy2022}.

Vervolgens worden de door Amplify aangeboden tools verder uitgelicht.

\subsubsection{Amplify libraries}

De Amplify open-source client libraries voorzien use-case gebaseerde, declaratieve en gemakkelijk te gebruiken interfaces
doorheen verschillende categorieën van cloud ondersteunende operaties.
Deze stellen mobiele en web ontwikkelaars in staat om integraties met hun backend infrastructuur te realiseren.
De libraries worden ondersteund door de AWS-cloud en bieden een inplugbaar model aan, dat ook door andere cloud leveranciers kan worden gebruikt.
Verder kunnen de libraries gebruikt worden met beide, nieuwe backend aangemaakt via de Amplify CLI en bestaande backend bronnen~\autocite{AWSAmplifyDocs}.

\subsubsection{Amplify Studio}

AWS Amplify Studio is een visuele ontwikkeling omgeving voor het bouwen van full-stack web en mobiele toepassingen.
Studio bouwt verder op bestaande backend-bouw mogelijkheden in AWS Simplify, UI-ontwikkeling wordt aanzienlijk versneld.
Met Studio is het mogelijk om een webtoepassing in zijn geheel te bouwen, front-to-back, met minimale codeer werk.
En toch behoudt de gebruiker volledige controle over de toepassing ontwerp en gedrag via code~\autocite{AWSAmplifyDocs}.

\subsubsection{Amplify CLI}

%The Amplify Command Line Interface (CLI) is a unified toolchain to create, integrate, and manage the AWS cloud services for your app.
De Amplify Command Line Interface (CLI) is een eendrachtig tool die het bouwen, integreren en beheren van AWS-services voor een toepassing faciliteert~\autocite{AWSAmplifyDocs}.
%TODO: verder uitleggen

\subsubsection{Amplify Hosting}

%Amplify Hosting provides a git-based workflow for hosting full-stack serverless web apps with continuous deployment.
%This user guide provides the information you need to get started with Amplify Hosting.

Amplify Hosting biedt een git-gebaseerde workflow aan voor het hosten van een full-stack serverless webtoepassing met continuous deployment(CD).
%TODO: verder uitleggen

\subsubsection{Marktevaluatie}

%https://financesonline.com/news/building-saas-on-aws-now-faster-with-ga-release-of-low-code-tool-amplify-studio/
%https://bejamas.io/discovery/hosting/aws-amplify/
%Amplify focuses on user experience and gives frontend developers one of the easiest ways to use and connect various AWS products.

In een AWS blog van \textcite{Spittel2022} legt hij uit hoe AWS Amplify Studio gemaakt is om het leven van ontwikkelaars gemakkelijker te maken.
De ontwerp-ontwikkelaar overdracht gebeurt vlotter d.m.v.\ Amplify.
Bovendien is de code aanpassen of uitbreiden heel eenvoudig met de gegenereerde componenten van Amplify Studio.

Ook \textcite{Nwamba2022} vertelt in zijn artikel hoe ontwikkelaars steeds meer low-code tools adopteren.
Ontwikkelaars die het ontwikkelwerk voor het bouwen van een toepassing interface willen vereenvoudigen moeten AWS Amplify overwegen.

\subsection{AWS Lightsail}
\label{subsec:service-lightsail}

AWS Lightsail verschaft cloud systeembronnen om een toepassing of webtoepassing moeiteloos te kunnen opzetten.
Lightsail biedt vereenvoudigde diensten aan zoals instanties, containers, databanken, opslag, enz.
Niet alleen is Lightsail compatibel met voorgeconfigureerde blueprints zoals Wordpress, Prestashop of LAMP\@.
Maar het is ook mogelijk om statische content te hosten met Lightsail en die globaal ter beschikking te stellen~\autocite{AWSLightsail2022}.

Verder worden een aantal kenmerken van Lightsail uitgelicht.

\begin{description}
    \item[Instances: ] Lightsail virtuele servers voor het hosten van toepassingen.
    \item[Containers: ] Deployen van Docker containers via Lightsail.
    \item[Simplified load balancers: ] Vereenvoudigde load balancing functionaliteit.
    \item[Managed databases: ] Volledig geconfigureerde MySQL of PostgreSQL databanken.
    \item[Block and object storage: ] Schaalbaar block en object opslag op de AWS-cloud.
    \item[CDN distributions: ] Content delivery networks (CDN) voor de globale distributie van toepassingen.
\end{description}

\subsubsection{Marktevaluatie}

In een blog van \textcite{Warrier2022} beschrijft hij Lightsail als een gemakkelijke tool om eenvoudige workloads te beheren
en voorgeconfigureerde infrastructuur, zoals toepassingen, databanken en andere populaire stacks, te provisioneren.

\subsection{AWS Copilot}
\label{subsec:service-copilot}

%AWS Copilot is an open-source command-line interface that makes it easy for developers to build,
%release, and operate production-ready containerized applications on AWS App Runner, Amazon ECS, and AWS Fargate.

The Copilot CLI is a tool for developers to build, release, and operate production-ready containerized applications on AWS App Runner,
Amazon ECS, and AWS Fargate.
From getting started, pushing to staging, and releasing to production, Copilot can help manage the entire lifecycle of your application development.

%Copilot makes it super easy to set up and deploy your containers on AWS - but getting started is only the first step of the journey.
%What happens when you want to have one copy of your service running only for testing and another copy serving production traffic?
%What happens when you want to add another service? How do you manage deploying to all of these services? Copilot wants to help you with
%all of these things so let's jump into some of Copilot's core concepts to understand how they can help.
~\autocite{Karakus2022}

AWS Copilot is een command line interface (CLI).
Deze tool~\autocite{Kumar2022}


\subsection{AWS Elastic Beanstalk}
\label{subsec:service-elastic-beanstalk}

\subsection{AWS Cloudformation}
\label{subsec:service-cloudformation}

Cloudformation is een AWS-service voor het modelleren en opzetten van AWS-resources.
Complexe opstellingen, waarbij meerdere AWS-resources van elkaar afhankelijk zijn en samen functioneren, kunnen als één geheel geconfigureerd worden.
Een Cloudformation template moet meerdere keren kunnen gebruikt worden.
Het is daarom ook mogelijk om via parameters de template dynamisch te maken.
De gebruiker kan a.d.h.v.\ deze parameters de configuratie van de AWS-resources anders instellen zonder de template elke keer te moeten aanpassen.
Voor infrastructuur en AWS-resources die via Cloudformation zijn opgezet bestaat de mogelijkheid om aanpassingen te brengen via change-sets.
Een alternatief is om bestaande infrastructuurcomponenten in een Cloudformation stack te opnemen.
Daarnaast biedt Cloudformation ook de functionaliteit aan om infrastructuur eenvoudig af te bouwen~\autocite{Mansoor2014}.

\subsubsection{Template}

Cloudformation werkt op basis van templates.
Een template is een JSON- of YAML-bestand waarin de gewenste AWS-resources en configuratie worden beschreven.
Deze templates moeten op een S3 bucket beschikbaar gesteld worden om door Cloudformation gebruikt te kunnen worden.
De mogelijkheid bestaat ook om naar een bestand op een lokaal systeem te verwijzen via de console, API of CLI.
AWS zorgt in dat geval dat het bestand wordt opgeladen op een S3 bucket.
Templates kunnen ook dynamisch geïmplementeerd worden door gebruik te maken van parameters.
Deze parameters worden gespecifieerd bij het aanmaken of actualiseren van een stack.
Zo wordt er vermeden om telkens opnieuw de template te moeten aanpassen~\autocite{AWSCLoudformationUser}.

\subsubsection{Stack}

Een Cloudformation stack is een collectie van AWS-resources die beheerd kunnen worden als één geheel.
De stack wordt aangemaakt op basis van een Cloudformation template.
Indien de template parameters bevat, dan kunnen die gespecifieerd worden bij het aanmaken van de stack.
Anders bestaat ook de mogelijkheid om default waarden in te stellen voor de parameters.
Nadat een stack is aangemaakt kan die geactualiseerd worden a.d.h.v.\ change-sets.
De aangemaakte resources kunnen eenvoudig afgebouwd worden door de stack te verwijderen~\autocite{AWSCLoudformationUser}.

\subsubsection{Change set}

De resources van een bestaande stack kunnen aangepast worden via een change set.
Daarvoor moet één of beide van de volgende acties uitgevoerd worden:

\begin{itemize}
    \item Aanpassingen brengen aan de bestaande template.
    \item Nieuwe input parameters specifiëren.
\end{itemize}

Een change set is de vergelijking van de huidige template t.o.v.\ de aangepaste versie.
In de change set worden de voorgestelde aanpassingen opgesomd.
Eens goedgekeurd worden de resources voor die stack bijgewerkt~\autocite{AWSCLoudformationUser}.

\subsubsection{Delete stack}

Bij het verwijderen van een stack worden de betreffende AWS-resources afgebouwd.
Het is mogelijk om bepaalde resources niet te verwijderen door een “deletion policy” te configureren waarbij de gewenste resources behouden blijven.
Indien één van de resources niet verwijderd kan worden dan kan de stack ook niet verwijderd worden.
In dat geval blijft de stack bestaan tot de systeemcomponenten succesvol verwijderd kunnen worden, of als alternatief kan aangegeven worden om deze te behouden buiten de stack~\autocite{AWSCLoudformationUser}.

\subsection{Terraform}
\label{subsec:service-terraform}

\subsection{Ansible}
\label{subsec:service-ansible}

\subsection{AWS SAM}
\label{subsec:service-sam}

\subsection{AWS CDK}
\label{subsec:service-cdk}

Cloud Development Kit (CDK) is een open source software development framework gebruikt voor het schrijven van Infrastructure as Code (IaC) in een voor de ontwikkelaar vertrouwde programmeertaal zoals:

\begin{itemize}
    \item TypeScript
    \item JavaScript
    \item Python
    \item Java
    \item C\#
\end{itemize}

AWS CDK gebruikt de voordelen van een programmeertaal om infrastructuur en toepassingen te schrijven, lezen en begrijpen~\autocite{Mansoor2014}.

De CDK-framework compileert de code en genereert een Cloudformation stack.
Daarom gelden dezelfde principes op deze tool als bij Cloudformation.
Via de CDK-Toolkit, een CLI-interface voor CDK, kunnen de standaard acties zoals een stack creëren, wijzigen en verwijderen uitgevoerd worden.
Daarnaast zijn troubleshooting functionaliteiten ook beschikbaar via de CDK-Toolkit~\autocite{Mansoor2014}.

\paragraph{App}

De App construct is de root klasse van alle constructs.
De klasse heeft geen argumenten nodig.
Bij het aanmaken van een Stack construct moet er verwezen naar een App construct.

\paragraph{Stack}

Een stack is de eenheid van deployment voor AWS CDK.
Een CDK-stack is gelijkwaardig aan een Cloudformation-stack en beschikt over dezelfde beperkingen.
Alle AWS-resources die gedefinieerd worden binnen een stack worden als één geheel beschouwd.

\paragraph{Construct}

Een construct ligt aan de basis van een CDK-app.
Het staat voor een AWS-resource en bevat alle configuratie die nodig is voor Cloudformation om de component te bouwen.
Een construct kan één AWS-resource vertegenwoordigen zoals een AWS S3 bucket of kan bestaan uit meerdere resources waarbij men spreekt over een higher-level abstraction construct.
De AWS CDK library bevat een collectie van constructs voor elke AWS-resource.
Er bestaan ook meer complexe third-party constructs die beschikbaar gesteld worden via de Construct Hub.

\paragraph{CDK-Toolkit}

Met behulp van de toolkit CLI kunnen alle functionaliteiten uitgevoerd worden die nodig zijn om de applicatie en infrastructuur te kunnen deployen en beheren zoals:

\begin{description}
    \item[Synthesis:] De synth commando compileert de code en genereert een Cloudformation template.
    \item[Deploy:] De deploy commando zal de stack deployen naar de geconfigureerde AWS-account, voor elke deploy wordt de code gesynthetiseerd.
    \item[Diff:] De diff commando vergelijkt de laatste stack met de laatste gesynthetiseerde template, de output is een overzicht van de verschillen tussen de twee.
    \item[Destroy:] De destroy commando verwijdert de stack en de bijhorende AWS-resources, dit is gelijkwaardig aan het verwijderen van een Cloudformation stack.
\end{description}

\subsection{Troposphere}
\label{sec:service-troposphere}